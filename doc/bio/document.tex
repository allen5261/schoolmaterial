\documentclass[11pt,a4paper,titlepage,twocolumn]{ctexart}
\usepackage{mhchem,extarrows}
\usepackage{amsmath}
\usepackage{amsfonts}
\usepackage{amssymb}
\usepackage{graphicx}
\usepackage{hyperref,multirow,bigstrut}
\usepackage[left=2.54cm, right=2.54cm, top=3.18cm, bottom=3.18cm]{geometry}
\title{\textbf{{\huge 模拟孟德尔的豌豆杂交实验}}}
\author{青岛第二十六中学\quad 二〇一五级一班\\王愉扬}
\date{\today}
\begin{document}
	\maketitle
	
	% Table generated by Excel2LaTeX from sheet 'Sheet1'
	\begin{tabular}{|c|c|c|c|c|c|}
		\hline
		\multirow{3}[2]{*}{实验名称} & \multirow{2}[1]{*}{模拟孟德尔的豌豆杂交实验} & \multicolumn{2}{c|}{\multirow{3}[2]{*}{实验人}} & \multicolumn{2}{c|}{\multirow{3}[2]{*}{青岛第二十六中学}} \bigstrut[t]\\
		&   & \multicolumn{2}{c|}{} & \multicolumn{2}{c|}{} \\
		&   & \multicolumn{2}{c|}{} & \multicolumn{2}{c|}{} \bigstrut[b]\\
		\hline
		\multicolumn{6}{|c|}{实验过程} \bigstrut\\
		\hline
		\multirow{4}[8]{*}{器材清单} & 名称 & \multicolumn{3}{c|}{规格} & 数量 \bigstrut\\
		\cline{2-6}  & 玻璃碗 & \multicolumn{3}{c|}{个} & 2 \bigstrut\\
		\cline{2-6}  & 白围棋子 & \multicolumn{3}{c|}{粒} & 20 \bigstrut\\
		\cline{2-6}  & 黑围棋子 & \multicolumn{3}{c|}{粒} & 20 \bigstrut\\
		\hline
		\multirow{6}[2]{*}{实验过程} & \multicolumn{5}{c|}{\multirow{6}[2]{30em}{1.   将20粒白围棋子(代表控制豌豆高茎的D基因)放入1号玻璃碗(代表亲代纯种高茎豌豆的精子)中,另外20粒黑围棋子(代表控制豌豆矮茎的d基因)放入2号玻璃碗(代表亲代纯种矮茎豌豆的卵细胞)中。}} \bigstrut[t]\\
		& \multicolumn{5}{c|}{} \\
		& \multicolumn{5}{c|}{} \\
		& \multicolumn{5}{c|}{} \\
		& \multicolumn{5}{c|}{} \\
		& \multicolumn{5}{c|}{} \bigstrut[b]\\
		\hline
		\multirow{2}[4]{*}{实验表格} & \multicolumn{2}{c|}{表格①} & \multicolumn{3}{r|}{} \bigstrut\\
		\cline{2-6}  & \multicolumn{2}{c|}{表格②} & \multicolumn{3}{r|}{} \bigstrut\\
		\hline
		\multirow{4}[2]{*}{实验结论} & \multicolumn{5}{c|}{\multirow{4}[2]{30em}{1.   纯种亲代杂交时,如图一,会产生基因型为Dd(性状为高茎)的子一代。}} \bigstrut[t]\\
		& \multicolumn{5}{c|}{} \\
		& \multicolumn{5}{c|}{} \\
		& \multicolumn{5}{c|}{} \bigstrut[b]\\
		\hline
		实验配图 & \multicolumn{5}{c|}{} \bigstrut\\
		\hline
	\end{tabular}%
	
	
\end{document}