\documentclass[titlepage]{article}
\usepackage{geometry} 		% 設定邊界
\geometry{
	top=2.54cm,
	inner=2.54cm,
	outer=3.18cm,
	bottom=3.18cm
}
\usepackage{ulem,indentfirst,amssymb}
\usepackage{xcolor}
\usepackage{graphicx,xeCJK}
\title{\textbf{如何有效地报告Bug\\\normalsize How to Report Bugs Effectively}}
\usepackage[setpagesize=false, % page size defined by xetex
unicode=false, % unicode breaks when used with xetex
xetex]{hyperref}
% commands generated by html2latex
\setCJKmainfont[BoldFont=思源黑体 Medium,ItalicFont=方正楷体_GBK]{思源宋体}
\setlength{\parindent}{2em}
\author{Simon Tatham $ \circledcirc $ 著\\Dasn $ \circledcirc $ 译}
\date{一九九九年\\[10em]\parbox{\linewidth}{版权所有 Simon Tatham 1999\\
		本文属于\href{https://www.opencontent.org/}{OPL(OpenContent License)},请在复制和使用本文时自觉遵守OPL。\\		
		对本文的任何意见和批评请发送至:\\	
		英文版:\href{mailto:anakin@pobox.com}{anakin@pobox.com}\\	
		中文版:\href{dasn@users.sf.net}{dasn@users.sf.net}}}
\hypersetup{
	bookmarks=true,colorlinks=true,allcolors=blue
}
\renewcommand{\contentsname}{\centering\LARGE 目\quad 录}
\begin{document}
\maketitle

\tableofcontents\newpage

\section{{引言}}

 为公众写过软件的人,大概都收到过很拙劣的bug(计算机程序代码中的错误或程序运行时的瑕疵——译者注)报告,例如:
%  I changed the style a little :) 


在报告中说“不好用”;

所报告内容毫无意义;

在报告中用户没有提供足够的信息;

在报告中提供了\emph{错误}信息;

所报告的问题是由于用户的过失而产生的;

所报告的问题是由于其他程序的错误而产生的;

所报告的问题是由于网络错误而产生的;

这便是为什么“技术支持”被认为是一件可怕的工作,因为有拙劣的bug报告需要处理。然而并不是所有的bug报告都令人生厌:我在业余时间维护自由软件,有时我会收到非常清晰、有帮助并且\emph{“有内容”}的bug报告。

在这里我会尽力阐明如何写一个好的bug报告。我非常希望每一个人在报告bug之前都读一下这篇短文,当然我也希望用户在给\emph{我}报告bug之前已经读过这篇文章。

简单地说,报告bug的目的是为了让程序员看到程序的错误。您可以亲自示范,也可以给出能导致程序出错的、详尽的操作步骤。如果程序出错了,程序员会收集额外的信息直到找到错误的原因;如果程序没有出错,那么他们会请您继续关注这个问题,收集相关的信息。

在bug报告里,要设法搞清什么是事实(例如:“我在电脑旁”和“XX出现了”)什么是推测(例如:“我\emph{想}问题可能是出在……”)。如果愿意的话,您可以省去推测,但是千万别省略事实。

当您报告bug的时候(既然您已经这么做了),一定是希望bug得到及时修正。所以此时针对程序员的任何过激或亵渎的言语(甚至谩骂)都是与事无补的——因为这可能是程序员的错误,也有可能是您的错误,也许您有权对他们发火,但是如果您能多提供一些有用的信息(而不是激愤之词)或许bug会被更快的修正。除此以外,请记住:如果是免费软件,作者提供给我们已经是出于好心,所以要是太多的人对他们无礼,他们可能就要“收起”这份好心了。

\section{{“程序不好用”}}

程序员不是弱智:如果程序一点都不好用,他们不可能不知道。他们不知道一定是因为程序在他们看来工作得很正常。所以,或者是您作过一些与他们不同的操作,或者是您的环境与他们不同。他们需要信息,报告bug也是为了提供信息。信息总是越多越好。

许多程序,特别是自由软件,会公布一个“已知bug列表”。如果您找到的bug在列表里已经有了,那就不必再报告了,但是如果您认为自己掌握的信息比列表中的丰富,那无论如何也要与程序员联系。您提供的信息可能会使他们更简单地修复bug。

本文中提到的都是一些指导方针,没有哪一条是必须恪守的准则。不同的程序员会喜欢不同形式的bug报告。如果程序附带了一套报告bug的准则,一定要读。如果它与本文中提到的规则相抵触,那么请以它为准。

如果您不是报告bug,而是寻求帮助,您应该说明您曾经到哪里找过答案,(例如:我看了第四章和第五章的第二节,但我找不到解决的办法。)这会使程序员了解用户喜欢到哪里去找答案,从而使程序员把帮助文档做得更容易使用。

\section{{“演示给我看”}}

报告bug的最好的方法之一是“演示”给程序员看。让程序员站在电脑前,运行他们的程序,指出程序的错误。让他们看着您启动电脑、运行程序、如何进行操作以及程序对您的输入有何反应。

他们对自己写的软件了如指掌,他们知道哪些地方不会出问题,而哪些地方最可能出问题。他们本能地知道应该注意什么。在程序真的出错之前,他们可能已经注意到某些地方不对劲,这些都会给他们一些线索。他们会观察程序测试中的每一个\emph{细节},并且选出他们认为有用的信息。

这些可能还不够。也许他们觉得还需要更多的信息,会请您重复刚才的操作。他们可能在这期间需要与您交流一下,以便在他们需要的时候让bug重新出现。他们可能会改变一些操作,看看这个错误的产生是个别问题还是相关的一类问题。如果您不走运,他们可能需要坐下来,拿出一堆开发工具,花上几个小时来\emph{好好地}研究一下。但是最重要的是在程序出错的时候让程序员在电脑旁。一旦他们看到了问题,他们通常会找到原因并开始试着修改。

\section{{“告诉我该怎么做”}}

如今是网络时代,是信息交流的时代。我可以点一下鼠标把自己的程序送到俄罗斯的某个朋友那里,当然他也可以用同样简单的方法给我一些建议。但是如果我的程序出了什么问题,我\emph{不可能}在他旁边。“演示”是很好的办法,但是常常做不到。

如果您必须报告bug,而此时程序员又不在您身边,那么您就要想办法让bug\emph{重现}在他们面前。当他们亲眼看到错误时,就能够进行处理了。

确切地告诉程序员您做了些什么。如果是一个图形界面程序,告诉他们您按了哪个按钮,依照什么顺序按的。如果是一个命令行程序,精确的告诉他们您键入了什么命令。您应该尽可能详细地提供您所键入的命令和程序的反应。

把您能想到的所有的输入方式都告诉程序员,如果程序要读取一个文件,您可能需要发一个文件的拷贝给他们。如果程序需要通过网络与另一台电脑通讯,您或许不能把那台电脑复制过去,但至少可以说一下电脑的类型和安装了哪些软件(如果可以的话)。

\section{{“哪儿出错了?在我看来一切正常哦!”}}

如果您给了程序员一长串输入和指令,他们执行以后没有出现错误,那是因为您没有给他们足够的信息,可能错误不是在每台计算机上都出现,您的系统可能和他们的在某些地方不一样。有时候程序的行为可能和您预想的不一样,这也许是误会,但是您会认为程序出错了,程序员却认为这是对的。

同样也要描述发生了什么。精确的描述您看到了什么。告诉他们为什么您觉得自己所看到的是错误的,最好再告诉他们,您认为自己应该看到什么。如果您只是说:“程序出错了”,那您很可能漏掉了非常重要的信息。

如果您看到了错误消息,一定要仔细、准确的告诉程序员,这\emph{确实}很重要。在这种情况下,程序员只要修正错误,而不用去找错误。他们需要知道是什么出问题了,系统所报的错误消息正好帮助了他们。如果您没有更好的方法记住这些消息,就把它们写下来。只报告“程序出了一个错”是毫无意义的,除非您把错误消息一块报上来。

特殊情况下,如果有错误消息号,\emph{一定}要把这些号码告诉程序员。不要以为您看不出任何意义,它就没有意义。错误消息号包含了能被程序员读懂的各种信息,并且很有可能包含重要的线索。给错误消息编号是因为用语言描述计算机错误常常令人费解。用这种方式告诉您错误的所在是一个最好的办法。

在这种情形下,程序员的排错工作会十分高效。他们不知道发生了什么,也不可能到现场去观察,所以他们一直在搜寻有价值的线索。错误消息、错误消息号以及一些莫名其妙的延迟,都是很重要的线索,就像办案时的指纹一样重要,保存好。

如果您使用UNIX系统,程序可能会产生一个内核输出(coredump)。内核输出是特别有用的线索来源,别扔了它们。另一方面,大多数程序员不喜欢收到含有大量内核输出文件的EMAIL,所以在发邮件之前最好先问一下。还有一点要注意:内核输出文件记录了完整的程序状态,也就是说任何秘密(可能当时程序正在处理一些私人信息或秘密数据)都可能包含在内核输出文件里。

\section{{“出了问题之后,我做了……”}}

当一个错误或bug发生的时候,您可能会做许多事情。但是大多数人会使事情变的更糟。我的一个朋友在学校里误删了她所有的Word文件,在找人帮忙之前她重装了Word,又运行了一遍碎片整理程序,这些操作对于恢复文件是毫无益处的,因为这些操作搞乱了磁盘的文件区块。恐怕在这个世界上没有一种反删除软件能恢复她的文件了。如果她不做任何操作,或许还有一线希望。

这种用户仿佛一只被逼到墙角的鼬(黄鼠狼、紫貂一类的动物——译者注):背靠墙壁,面对死亡的降临奋起反扑,疯狂攻击。他们认为做点什么总比什么都不做强。然而这些在处理计算机软件问题时并不适用。

不要做鼬,做一只羚羊\footnote{\emph{声明:}我从没有真的看见过鼬和羚羊,我的比喻可能不恰当。}。当一只羚羊面对料想不到的情况或受到惊吓时,它会一动不动,是为了不吸引任何注意,与此同时也在思考解决问题的最好办法(如果羚羊有一条技术支持热线,此时占线。)。然后,一旦它找到了最安全的行动方案,它便去做。

当程序出毛病的时候,立刻停止正在做的\emph{任何操作}。不要按任何健。仔细地看一下屏幕,注意那些不正常的地方,记住它或者写下来。然后慎重地点击“确定” 或“取消”,选择一个最安全的。学着养成一种条件反射——一旦电脑出了问题,先不要动。要想摆脱这个问题,关掉受影响的程序或者重新启动计算机都不好,一个解决问题的好办法是让问题再次产生。程序员们喜欢可以被重现的问题,快乐的程序员可以更快而且更有效率的修复bug。

\section{{“我想粒子的跃迁与错误的极化有关”}}

并不只是非专业的用户才会写出拙劣的bug报告,我见过一些非常差的bug报告出自程序员之手,有些还是非常优秀的程序员。

有一次我与另一个程序员一起工作,他一直在找代码中的bug,他常常遇到一个bug,但是不会解决,于是就叫我帮忙。“出什么毛病了?”我问。而他的回答却总是一些关于bug的意见。如果他的观点正确,那的确是一件好事。这意味着他已经完成了工作的一半,并且我们可以一起完成另一半工作。这是有效率并有用的。

但事实上他常常是错的。这就会使我们花上半个小时在原本正确的代码里来回寻找错误,而实际上问题出在别的地方。我敢肯定他不会对医生这么做。“大夫,我得了Hydroyoyodyne(真是怪病——译者),给我开个方子”,人们知道不该对一位医生说这些。您描述一下症状,哪个地方不舒服,哪里疼、起皮疹、发烧……让医生诊断您得了什么病,应该怎样治疗。否则医生会把您当做疑心病或精神病患者打发了,这似乎没什么不对。

做程序员也是一样。即便您自己的“诊断”有时真的有帮助,也要只说“症状”。“诊断”是可说可不说的,但是“症状”一定要说。同样,在bug报告里面附上一份针对bug而做出修改的源代码是有用处的,但它并不能替代bug报告本身。

如果程序员向您询问额外的信息,千万别应付。曾经有一个人向我报告bug,我让他试一个命令,我知道这个命令不好用,但我是要看看程序会返回一个什么错误(这是很重要的线索)。但是这位老兄根本就没试,他在回复中说“那肯定不好用”,于是我又花了好些时间才说服他试了一下那个命令。

用户多动动脑筋对程序员的工作是有帮助的。即使您的推断是错误的,程序员也应该感谢您,至少您\emph{想}去帮助他们,使他们的工作变的更简单。不过千万别忘了报告“症状”,否则只会使事情变得更糟。

\section{{“真是奇怪,刚才还不好用,怎么现在又好了?”}}

“间歇性错误”着实让程序员发愁。相比之下,进行一系列简单的操作便能导致错误发生的问题是简单的。程序员可以在一个便于观察的条件下重复那些操作,观察每一个细节。太多的问题在这种情况下不能解决,例如:程序每星期出一次错,或者偶然出一次错,或者在程序员面前从不出错(程序员一离开就出错。——译者)。当然还有就是程序的截止日期到了,那肯定要出错。

大多数“间歇性错误”并不是真正的“间歇”。其中的大多数错误与某些地方是有联系的。有一些错误可能是内存泄漏产生的,有一些可能是别的程序在不恰当的时候修改某个重要文件造成的,还有一些可能发生在每一个小时的前半个小时中(我确实遇到过这种事情)。

同样,如果您能使bug重现,而程序员不能,那很有可能是他们的计算机和您的计算机在某些地方是不同的,这种不同引起了问题。我曾写过一个程序,它的窗口可以\emph{蜷缩}成一个小球呆在屏幕的左上角,它在别的计算机上只能在 800x600 的解析度工作,但是在我的机器上却可以在 1024x768 下工作。

程序员想要了解任何与您发现的问题相关的事情。有可能的话您到另一台机器上试试,多试几次,两次,三次,看看问题是不是经常发生。如果问题出现在您进行了一系列操作之后,不是您想让它出现它就会出现,这就有可能是长时间的运行或处理大文件所导致的错误。程序崩溃的时候,您要尽可能的记住您都做了些什么,并且如果您看到任何图形,也别忘了提一下。您提供的任何事情都是有帮助的。即使只是概括性的描述(例如:当后台有EMACS运行时,程序常常出错),这虽然不能提供导致问题的直接线索,但是可能帮助程序员重现问题。

最重要的是:程序员想要确定他们正在处理的是一个真正的“间歇性错误”呢,还是一个在另一类特定的计算机上才出现的错误。他们想知道有关您计算机的许多细节,以便了解您的机器与他们的有什么不同。有许多细节都依仗特定的程序,但是有一件东西您一定要提供——版本号。程序的版本、操作系统的版本以及与问题有关的程序的版本。

\section{{“我把磁盘装进了 Windows……”}}

表意清楚在一份bug报告里是最基本的要求。如果程序员不知道您说的是什么意思,那您就跟没说一样。我收到的bug报告来自世界各地,有许多是来自非英语国家,他们通常为自己的英文不好而表示歉意。总的来说,这些用户发来的bug报告通常是清晰而且有用的。几乎所有不清晰的bug报告都是来自母语是英语的人,他们总是以为只要自己随便说说,程序员就能明白。
\begin{itemize}
	\item \textbf{精确}。如果做相同的事情有两种方法,请说明您用的是哪一种。例如:“我选择了‘载入’”,可能意味着“我用鼠标点击‘载入’”或“我按下了‘ALT+L’”,说清楚您用了哪种方法,有时候这也有关系。
	\item \textbf{详细}。信息宁多毋少!如果您说了很多,程序员可以略去一部分,可是如果您说的太少,他们就不得不回过头再去问您一些问题。有一次我收到了一份bug报告只有一句话,每一次我问他更多事情时,他每次的回复都是一句话,于是我花了几个星期的时间才得到了有用的信息。
	\item \textbf{慎用代词}。诸如“它”,“窗体”这些词,当它们指代不清晰的时候不要用。来看看这句话:“我运行了FooApp,它弹出一个警告窗口,我试着关掉它,它就崩溃了。”这种表述并不清晰,用户究竟关掉了哪个窗口?是警告窗口还是整个FooApp程序?您可以这样说,“我运行FooApp程序时弹出一个警告窗口,我试着关闭警告窗口,FooApp崩溃了。”这样虽然罗嗦点,但是很清晰不容易产生误解。
	\item \textbf{检查}。重新读一遍您写的bug报告,\emph{您}觉得它是否清晰?如果您列出了一系列能导致程序出错的操作,那么照着做一遍,看看您是不是漏写了一步。
\end{itemize}

\section{{小结}}
\begin{itemize}
	\item bug报告的首要目的是让程序员亲眼看到错误。如果您不能亲自做给他们看,给他们能使程序出错的详细的操作步骤。
	\item 如果首要目的不能达成,程序员\emph{不能}看到程序出错。这就需要bug报告的第二个目的来描述程序的什么地方出毛病了。详细的描述每一件事情:您看到了什么,您想看到什么,把错误消息记下来,\emph{尤其}是“错误消息号”。
	\item 当您的计算机做了什么您料想不到的事,\emph{不要动}!在您平静下来之前什么都别做。不要做您认为不安全的事。
	\item 尽量试着自己“诊断”程序出错的原因(如果您认为自己可以的话)。即使做出了“诊断”,您仍然应该报告“症状”。
	\item 如果程序员需要,请准备好额外的信息。如果他们不需要,就不会问您要。他们不会故意为难自己。您手头上一定要有程序的版本号,它很可能是必需品。
	\item 表述清楚,确保您的意思不能被曲解。
	\item 总的来说,最重要的是要做到\emph{精确}。程序员喜欢精确。
\end{itemize}

\end{document}
