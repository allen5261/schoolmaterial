\documentclass[10pt,a4paper]{article}
\usepackage{geometry} 		% 設定邊界
\geometry{
	top=2.54cm,
	inner=2.54cm,
	outer=3.18cm,
	bottom=3.18cm
}
\usepackage{ulem,indentfirst,amssymb}
\usepackage{xcolor}
\usepackage{graphicx,xeCJK}
\title{\textbf{洛谷\quad 帮助}}
\date{}
\usepackage[setpagesize=false, % page size defined by xetex
unicode=false, % unicode breaks when used with xetex
xetex]{hyperref}
% commands generated by html2latex
\setCJKmainfont[BoldFont=思源黑体 Medium,ItalicFont=方正楷体_GBK]{思源宋体}
\setlength{\parindent}{2em}
\hypersetup{
	bookmarks=true,colorlinks=true,allcolors=blue
}
\renewcommand{\contentsname}{\centering\LARGE 目\quad 录}
\begin{document}
	\maketitle\tableofcontents\newpage
	
	
	洛谷创建于2013年6月15日,至今已有数万用户,致力于为OIers/ACMers提供清爽、快捷的编程体验。它不仅仅是一个在线测题系统,更拥有强大的社区、在线学习功能。同时,许多教程内容都是由五湖四海的OIers提供的,保证了内容的广泛性。无论是初学oi的蒟蒻,还是久经沙场的神犇,均可从洛谷Online 
	Judge获益,也可以帮助他人,共同进步。
	
	\section{
		评测}
	
	
	洛谷评测系统搭建于Linux上,采用分布式集群保证评测效率,采用沙盒技术保证评测安全。目前,评测系统支持四种语言:C/C++/C++11/Pascal。其编译参数分别为:
	\begin{verbatim}
	
	- C:gcc -DONLINE_JUDGE -Wall -fno-asm -std=c99 -lm
	- C++:g++ -DONLINE_JUDGE -Wall -fno-asm -std=c++98 
	- C++11:g++ -DONLINE_JUDGE -Wall -fno-asm -std=c++11
	- Pascal:ppcx64 -dONLINE_JUDGE\end{verbatim}
	
	
	在大牛模式下进行提交的所有题目或是有"O2优化"标签的题目在评测时均会自动开启O2优化,题目上传者或者管理员可根据需要自行开启。
	
	\section{
		各个评测状态}
	\begin{itemize}{
			\item 
			
			AC:Accept,程序通过。
			\item 
			
			CE:Compile 
			Error,编译错误。
			\item 
			
			PC:Partially 
			Correct,部分正确。
			\item 
			
			WA:Wrong 
			Answer,答案错误。
			\item 
			
			RE:Runtime 
			Error,运行时错误。
			\item 
			
			TLE:Time Limit 
			Exceeded,超出时间限制。
			\item 
			
			MLE:Memory Limit 
			Exceeded,超出内存限制。
			\item 
			
			OLE:Output Limit 
			Exceeded,输出超过限制。
			\item UKE:Unknown Error,出现未知错误。}
	\end{itemize}
	
	\section{
		常见“我在本地/xxOJ AC了、洛谷却不过”的原因}
	\begin{itemize}{
			\item 
			
			
			Linux中换行符是'$\backslash$n'而Windows中是'$\backslash$r$\backslash$n'(多一个字符),有些数据在Windows中生成,而在洛谷评测机Linux环境下评测。这种情况在字符串输入中非常常见。
			\item 
			
			
			评测系统建立在Linux下,可能由于使用了Linux的保留字而出现CE,但在Windows下正常。
			\item 
			
			
			Linux对内存的访问控制更为严格,因此在Windows上可能正常运行的无效指针或数组下标访问越界,在评测系统上无法运行。
			\item 
			
			
			严重的内存泄露的问题很可能会引起系统的保护模块杀死你的进程。因此,凡是使用
			\texttt{malloc}(或
			\texttt{calloc,realloc,new})分配而得的内存空间,请使用
			\texttt{free}(或
			\texttt{delete})完全释放。
			\item 数据可能真的有问题。但是如果不止一个人通过了这道题,那最好不要怀疑是数据的锅。}
	\end{itemize}
	
	\subsection{
		Special Judge}
	
	\href{https://www.luogu.org/wiki/show?name=%E5%B8%AE%E5%8A%A9%EF%BC%9Aspecial%20judge}{
		帮助:special judge}
	
	\section{
		用户}
	
	\subsection{
		账户}
	
	
	任何用户必须遵守洛谷用户协议和洛谷社区规则,方可在站内进行学习交流。违反规则的将按照相关条令进行处理。
	\begin{itemize}{
			\item 
			
			
			注册:注册时你需要提供一个合法邮箱。\href{https://www.luogu.org/app/exception}{点击这里}注册一个洛谷账号
			\item 
			
			
			忘记密码:当你忘记密码时,可以通过注册邮箱找回账号。\href{https://www.luogu.org/app/exception}{点击这里}找回密码。如果你忘记了邮箱,可以向管理员申诉。
			\item 
			申诉:当你的账户有异常行为或者被认为存在安全问题时,洛谷会对账号进行冻结。冻结的账号可以解封。你可以进行申诉。}
	\end{itemize}
	
	\textbf{{
			等级}}
	
	
	洛谷的等级由积分决定。
    \begin{verbatim}
	0~10:蒟蒻  
	11~30:小小牛  
	31~60:小小犇
	61~100:小牛
	101~180:小犇
	181~280:中牛
	281~460:中犇
	461~740:大牛
	741~1200:大犇
	1201~1980:神牛
	1981~INF:神犇\end{verbatim}
	
	\textbf{{
			动态等级与用户名颜色}}
	
	
	动态等级是根据用户在一段时间内的刷题、社区活跃、打卡、题解、违规情况,按照一定算法得出的结果,该算法不公开。动态等级比较准确的表现了用户一段时间内在洛谷的综合行为。动态等级由-1到4,用户名的颜色为分别对应为棕色、灰色、蓝色、绿色、橙色、红色。管理员的用户名颜色为紫色,不受动态等级影响。
	\begin{verbatim}
	
	-1 棕色 作弊者
	0 灰色 见习用户 
	1 蓝色 普通用户 
	2 绿色 算法爱好者
	3 橙色 刷题健将
	4 红色 虐题狂魔\end{verbatim}
	
	
	新用户的初始动态等级为0,也就是灰色用户名。用户在洛谷中按照洛谷社区规则使用各项功能,会增动态等级。如果长时间不使用洛谷或者有违规行为,可能会降级。除非用户有严重违规行为,一旦用户达到蓝名,则不会掉到灰名。你可以在打卡之后在打卡结果中看到自己用户名的颜色。\textbf{{在犇犇或者讨论中询问颜色名字相关的一些问题将被和谐。}}如果用户抄袭题解或非恶意比赛作弊,将降级至棕名及名字旁带有Cheater标签,每次查到持续15日,并且在90日内不得在任何比赛获得排名。15日后,变成灰名并清空所有AC记录。(即,变为Unaccepted 
	100分状态)。(注:\textbf{{棕名用户是仅次于封号的最严厉警告,如果发生任何违反洛谷社区规则的事情,直接立刻封禁一年。}})
	
	
	动态等级影响洛谷内很多权限,例如创建比赛、团队、提交冷却时间、图床、下载数据等。灰名会有较多的限制。
	
	\subsection{
		私信}
	
	
	用户可以通过私信对其他用户留言,私信的内容不会被别人看到。私信暂时不能做到实时聊天,若要聊天请手动刷新。
	
	
	聊天输入框中不建议发送代码,当需要发送代码或者其他长文字时,在发送按钮的下方有一个代码按钮,在里面可以粘贴长文本然后发送。
	
	\subsection{
		通知}
	
	
	系统的通知将会出现在通知里面。包括:其他用户的@、题解审核结果、题目审核结果、举报审核结果、升级通知和其他的通知。
	
	\section{
		比赛}
	
	
	洛谷拥有强大的比赛功能,可以模拟进行各类比赛。比赛的公开度有以下几种:
	\begin{itemize}{
			\item 
			
			
			官方比赛:洛谷官方出题的比赛。例如洛谷月赛,题目均由管理员仔细审核或者干脆自己命题,质量有保证,而且一般优胜者都有奖品。有时候也会有一些测试新功能或者娱乐性质的比赛。在首页上展示。
			\item 
			
			
			个人公开赛:也经过管理员认可,题目质量较高,可能会有官方赞助的奖品。在首页上展示。
			\item 
			
			
			团队公开赛:以团队的名义建立,其他同个人公开赛。
			\item 
			
			
			个人邀请赛:用户自由的上传比赛,没有经过管理员审核,质量可能参差不齐。需要邀请码。
			\item 团队内部赛:如字面上的意思。}
	\end{itemize}
	
	
	洛谷提供丰富多彩的形式
	\begin{itemize}{
			\item 
			
			
			OI赛制:传统的赛制,比赛期间不能看到结果。以最后一次提交为准。
			\item 
			
			
			乐多赛制:洛谷独创的赛制。结合了OI赛制和ACM赛制的优点,既可以按照题目的测试点分点得分,也不失比赛的刺激。比赛时可以看到结果。对于一道题的得分,计算为(评测得分*0.95\textasciicircum(本题提交次数-1)),最低扣到原来的70\%为止。
			\item 
			
			
			ACM赛制:比赛时可以看到结果,必须AC了这道题目才会计分,会记录AC这道题的耗时,每次失败的提交会加上20分钟的罚时。通过题目数多的排名在前;通过数一样的耗时少排名靠前。
			\item IOI赛制:最不刺激的赛制,比赛时可以看到结果,计分按照这道题目的最高得分。}
	\end{itemize}
	
	\section{
		个人题库与个人比赛}
	
	
	只要你是洛谷用户并达到一定等级,你就有权限\href{https://www.luogu.org/app/userproblem}{创建私有题目}、\href{https://www.luogu.org/app/userproblem}{创建个人比赛}等。
	
	\subsection{
		个人题库与上传}
	
	
	进入\href{https://www.luogu.org/app/userproblem}{这里}之后,点击新建题目,填写需要的内容(可以不都填)之后保存题面。可以使用markdown美化详见\href{https://www.luogu.org/wiki/show?name=%E5%B8%AE%E5%8A%A9%EF%BC%9Amarkdown}{帮助:markdown}。再上传数据包即可。
		
		
		上传压缩包的要求:
		\begin{itemize}{
				\item 
				
				
				直接将若干数据点打包成一个zip压缩包,rar和其他格式不能成功。
				\item 
				
				
				没有任何文件夹或者其他无关文件,压缩后大小不超过10M。
				\item 
				测试点文件名中只能允许有连续的一段数字,例如'game001.in'可以,而'T1-1.in'或'game.in'不可以。}
		\end{itemize}
		
		\subsection{
			举办比赛}
		
		
		进入\href{https://www.luogu.org/app/userproblem}{这里}之后,点击新建比赛,填写需要的内容之后完成比赛设置即可。个人比赛的创建者可以看到所有提交者的代码。
		
		\textbf{{
				举办比赛要求:}}
		\begin{itemize}{
				\item 
				
				\textbf{{
						绿名及以上才可以举办个人公开赛}}。公开赛将会显示在洛谷首页。对于优秀的公开赛,洛谷将会为比赛赞助奖品。
				\item 
				对于过水、无数据、一直延时的比赛,将面临删除危险。若情况严重,比赛创建者可能会被冻结账号。}
		\end{itemize}
		
		\textbf{{
				公开赛要求:}}
		
		
		公开赛包括:个人公开赛、团队公开赛,审核标准一致。
		
		
		为了维持公开比赛的高质量,特制定以下规则:
		
		
		以下比赛属于不合格比赛,将会被爆破:
		
		
		1、题目过水(过难不管)、过少。包括但不限于普及组前两题难度、经典题的比赛。
		
		
		2、一拖再拖长期霸占版面的比赛。
		
		
		3、没有数据的比赛以及原创题没有标程测试通过的比赛。包括但不限于不可做的原创题而只能靠打表输出的题目。
		
		
		4、包括洛谷已经有的题目或者其他经典题目,或者套用现成的题目,只是进行一些微小改动的而在算法上基本没有实质不同的。
		
		
		5、可能影响洛谷正常评测秩序的比赛,例如猜随机数。
		
		
		请各位出题人自己对照以上几条标准判断自己的比赛是否满足要求,不满足要求可以改为邀请赛,在管理员发现之前不受惩罚。
		
		\section{
			恶意创建公开比赛属于II类违反,一次即可封号。}
		
		
		个人邀请赛和团队内部赛比较宽松,但不可以违反以下几点
		
		
		1、可能影响洛谷正常评测秩序的比赛,例如猜随机数。
		
		
		2、带有侮辱性质的比赛。
		
		\section{
			题目}
		
		
		题目系统是洛谷Online Judge的核心,要开始写一道新的题目,一般就是从这里开始的。
		
		\subsection{
			创建}
		
		
		见“个人题库与上传”。
		
		\subsection{
			提交}
		
		
		当用户没有登录,或者没有报名该题目所在的正在进行中的比赛,将无法提交题目。评测系统使用控制台标准输入输出,即提交的程序无需进行文件操作。评测忽略行末空格与文尾回车。不允许手动开O2。
		
		\subsection{
			题解}
		
		
		题目不会做时,可以学习题解,但是过度的抄袭题解代码被发现将会收到惩罚。认为题目很有价值时,若题解很少或有与现有题解都不一样的方法时,可以上传题解。上传题解需注意:
		\begin{itemize}{
				\item 
				
				不要上传与之前题解思路一样的做法。
				\item 
				
				
				不要上传纯代码而没有说明。但是可以纯说明没有代码(前提是可以让人看懂)。
				\item 
				
				
				可以在题解中给自己的blog打广告,但是不可以只有blog链接,必须有原文。除此之外不欢迎任何广告。
				\item 
				
				题解必须是原创,不得抄袭。
				\item 若违反以上内容,将会受到轻则减少积分,重则冻结账号的惩罚。}
		\end{itemize}
		
		\subsection{
			下载测试数据}
		
		
		只要评测完毕,而且至少有一个点没有通过,就可以下载第一个错误的点的输入输出数据。不过,过于依赖数据,会减弱自己程序的调试能力,不利于能力的培养。因此洛谷限制了每天下载个数。每位用户每天可以下载的次数即为自己的动态等级。望合理利用该功能,不要滥用。
		
		
		部分题目因为版权的问题不提供测试数据,部分过于大的测试点,也不会提供。
		
		\subsection{
			代码公开计划}
		
		
		代码公开计划是洛谷Online Judge创新性的的写题协助计划,允许用户在一定条件下查看他人的代码,吸取他人长处,获得解题思路,得到进步。
		
		\textbf{{
				如何查看他人代码}}
		\begin{itemize}{
				\item 
				
				
				当用户的某道题达到60分,且已加入代码公开计划,就可以查看其他加入代码公开计划的用户这道题的代码。
				\item 
				
				
				如果某条记录的提交者加入了的“源码公开计划”,点击可进入“记录详情页面”,查看该用户代码。
				\item 比赛代码不可查看。}
		\end{itemize}
		
		\textbf{{
				如何加入、退出代码公开计划}}
		\begin{itemize}{
				\item 
				
				
				对于每道题目,所有用户均默认加入计划。
				\item 
				
				
				如果用户针对某题不愿意加入该计划,可以在信息修改设置,即可退出。
				\item 
				用户取消了某题的代码公开计划,只需在信息修改设置即可加入。然而,为了保证用户代码能够展示一段时间,再次取消该计划需要等待一段“冷却期”,目前冷却期为15天。}
		\end{itemize}
		
		\section{
			试练场}
		
		
		为了方便不同水平的同学们都能在洛谷快速找到适合自己的训练方式,洛谷邀请了多位NOI大神精挑细选定制了各种难度、各种类型的专题,以打怪通关的形式,边玩边学。
		
		\subsection{
			规则}
		\begin{itemize}{
				\item 
				
				
				部分专题有先决要求,必须通过指定的专题才能开启这些专题。
				\item 
				
				
				每个专题只需要完成部分题目即可通过,要完成多少题视专题而定。
				\item 
				
				
				如果通过专题困难,可以跳过专题,一个人有3次跳过机会。跳过后可以回头完成这些跳过专题,通过后,可跳过的机会的次数可以恢复。
				\item 如果有改进的建议,可以向管理员提出。}
		\end{itemize}
		
		\section{
			团队}
		
		
		在洛谷Online Judge,团队不再是简单的聚合体,而是多功能的集合型圈子。
		
		\subsection{
			种类}
		
		
		团队分为以下种类:
		\begin{itemize}{
				\item 
				
				
				私有团队。此类团队不在列表中显示,也不具有出公开赛的权限。\textbf{{此类团队将严禁在洛谷进行任何形式的宣传,包括犇犇、讨论、私信群发等。}}但不禁止在其他平台进行宣传。新团队建立即为此权限,建议需要私有化培训、不希望公开的团队使用。
				\item 
				
				
				公开团队。此类团队在“公开团队”列表显示,但不具备出公开赛的权限。此类型团队将在下一版本洛谷更新中被移除并全部转回私有团队。
				\item 
				
				
				学校团队。建立此类团队请以学校名字命名,当有5人以上成员时,管理员将允许其显示在“学校团队”列表(目前尚未开辟,稍后将加入)中,并不再默认给予公开赛权限,有需要请发帖申请。\textbf{{注意:如果发现自己学校被他人冒充建立团队,请立刻举报。被发现假冒其他学校创建团队,视情节严重一次即可处以封号甚至封禁IP一年的处罚。}}
				\item 比赛团队。此类团队与公开团队相同,但允许出公开赛。申请时需要同时带上优质题目。}
		\end{itemize}
		
		\subsection{
			团队宣传规定}
		
		
		1、本规定最终解释权归洛谷管理组所有、并可能在不预先通知的情况下变更。
		
		
		2、公开团队允许每3日在团队宣传版发帖进行宣传,但每一页同时只能出现同一个团队的一条宣传帖。
		
		
		3、\textbf{{多条团队宣传帖的内容严禁相同。}}即,每次宣传必须撰写新的宣传稿。简单的从以前的复制粘贴会被删除。
		
		
		4、比赛团队还可在团队宣传版宣传比赛,频率次数规定相同。
		
		
		5、允许\textbf{{比赛团队}}举办的公开赛在犇犇进行3次、在首页讨论版进行1次的宣传。在团队宣传版进行的宣传,按照原有的规定继续执行。
		
		
		6、学校团队原则上\textbf{{不允许宣传}}。因为学校团队应该是学校内部使用。唯一的例外是学校团队出公开赛时,允许进行宣传。
		
		
		7、私密团队\textbf{{禁止宣传}}。
		
		
		8、公开团队举办的邀请赛,允许在犇犇进行1次宣传,禁止在首页讨论版进行宣传。在团队宣传版进行的宣传,按照原有的规定继续执行。
		
		
		9、当一个团队的宣传过度时,将会直接处以封禁团队处罚,除非负责人愿意写300字以上检查,说明对洛谷造成的影响、今后的改进措施等予以解封。\textbf{{洛谷的任何一名秩序管理员都有权无理由认为一个团队的宣传过度或者不当,即使其没有超过宣传次数的限制。我们保证将以合理、不带偏见的方式审核所有团队的宣传,但也请部分团队在宣传时自重}}。
		
		\subsection{
			创建}
		
		
		团队允许自由创建,同时洛谷鼓励大家组建各类同好会、学习小组、兴趣组等。
		
		
		创建后的团队默认为私有团队,不在团队列表中显示,各种功能的使用都有一定的限制且无法创建团队公开赛。若需开通以上功能,请联系管理员进行审核。
		
		
		对于学校类型的团队,我们将定期查看并对具有一定规模的活跃学校团队自动开通以上功能。其他团队用户请主动联系管理员,并说明理由。
		
		\subsection{
			加入}
		
		
		当自己决定要加入一个团队时,可以进入团队列表,并搜索一个团队。
		
		
		对于未公开的团队,是无法通过搜索等方式进入的,请通过团队内成员分享的团队链接加入团队。
		
		\subsection{
			团队详情}
		
		\textbf{{
				团队宣言}}
		
		
		公开展示的地方。在这里可以宣扬自己团队的风采。
		
		\textbf{{
				团队公告}}
		
		
		只有团队成员才能看到。用来布置题目、发布训练安排再好不过了。
		
		\textbf{{
				团队成员}}
		
		
		列出所有团队成员。管理员高亮显示。如果团队成员在“信息修改”中填写了真实姓名,将会这里备注出来,当然只有团队的成员才能看到。
		
		\subsection{
			作业}
		
		
		团队内的成员可以随时查看成员的完成情况,点击作业标题即可查看。已经按照作业的完成度对作业内的组员进行排序。对于一个成员和一个题目显示的是提交后最高分数。管理员可以布置题目,还能查看成员代码。作业的题目可以来自团队题目或者洛谷公开题目。
		
		\section{
			讨论}
		
		\subsection{
			板块}
		
		
		讨论区共分为四大板块
		\begin{itemize}{
				\item 
				
				
				站务版:管理团队将会正在这里发布网站重要信息。勿水。
				\item 
				
				
				题目总版:你可以在这里提出关于洛谷中题目的各种问题。
				\item 
				
				
				学术版:你可以在这里提出其他学术问题。
				\item 灌水区:为所有有话想说的同学们敞开大门。}
		\end{itemize}
		
		
		除此之外,每一道题目都有一个专属板块,可以从题目详情页面进入,也可以在讨论区主页右侧输入题号进入。
		
		\subsection{
			发帖}
		
		
		在讨论区主页选择板块后,在页面下方可以发帖。在每一帖子页面下方可以进行回帖。帖子可以用Markdown进行美化和@其他用户。详见\href{https://www.luogu.org/wiki/show?name=%E5%B8%AE%E5%8A%A9%EF%BC%9Amarkdown}{帮助:markdown}。
			
			\section{
				图床}
			
			
			图床用于上传本地图片。如果题目、讨论、签名需要插入图片时,请将图片先上传至图床,再使用Markdown代码引用图片在图床中的编号即可。
			
			
			图床限制图片的文件大小为500kb,如果上传的图片尺寸太大,则会对图片尺寸进行压缩。
			
			
			注意事项
			\begin{itemize}{
					\item 
					
					
					原则上只允许上传与信息学有关的图片,当然每个用户也可以上传少量自己的签名图等。(有关题目的可以上交多张)
					\item 不能上传违反国家法律与社会公德的图片。}
			\end{itemize}
			
			
			如有违反,视情节予以一定的处罚。
			
			\section{
				捐助我们}
			
			
			详见:\href{https://www.luogu.org/discuss/show?postid=17348}{给洛谷投食}
			
			\section{
				联系我们}
			
			\href{https://www.luogu.org/wiki/show?name=%E8%81%94%E7%B3%BB%E6%88%91%E4%BB%AC}{
				联系我们}
\end{document}